\section{Introduction}

\begin{frame}{Slide-Making in \LaTeX}

	We assume that you can use \LaTeX. If not, you can refer to \href{https://www.overleaf.com/learn/latex/Learn_LaTeX_in_30_minutes}{this page}.

	\href{https://www.overleaf.com/learn/latex/Beamer}{Beamer} is one of the most popular and influential document classes for slide-making in \LaTeX. You can find its \href{https://mirror-hk.koddos.net/CTAN/macros/latex/contrib/beamer/doc/beameruserguide.pdf}{full manual here}.

	Here, we will only introduce the basic functionalities so you can master them immediately.
\end{frame}


\begin{frame}{Beamer vs. MS PowerPoint}

	Compared to Microsoft PowerPoint, \LaTeX\ and Beamer provides these advantages:
	
	\begin{itemize}
		\item Beamer produces a \texttt{.pdf} file with no problems on fonts, formulas, or program versions.
		\item Math typesetting in \LaTeX\ is much easier, e.g.,
			\begin{equation*}
				\mathrm{i}\,\hslash\frac{\partial}{\partial t} \Psi(\mathbf{r},t) =
				-\frac{\hslash^2}{2\,m}\nabla^2\Psi(\mathbf{r},t)
				+ V(\mathbf{r})\Psi(\mathbf{r},t).
			\end{equation*}
	\end{itemize}
\end{frame}